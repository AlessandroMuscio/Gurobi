% Il documento è un article, LaTeX ha delle classi standard che impostano già dei parametri per te, in più gli ho aumentato la grandezza al font così arriviamo prima ad una pagina ;)
\documentclass[11pt]{article}

% Imposto la codifica del file ad UTF-8
\usepackage[utf8]{inputenc}
% Imposto la lingua in italiano
\usepackage[italian]{babel}
% Questo ho capito che imposta la codifica del font ma non so cosa sia T1
\usepackage[T1]{fontenc}
% Imposto il formato, dimensioni totali, margini 
\usepackage[a4paper,total={180mm, 267mm},left=14mm,top=5mm]{geometry}
% I prime tre pacchetti ti fanno fare tutte le equazione super belle con i simboli belli, il 4° vedi dopo
\usepackage{amsfonts, amsmath, amssymb, fancyhdr}

% Le impostazioni di indentazione di default di LaTeX sono strane, questo le rimuove
\setlength{\parindent}{0pt}

% Queste 7 righe sono possibili grazie al famoso 4° pacchetto, LaTeX di default mi mette delle barre nere a inizio e a pie di pagina inoltre a mettere il numero di pagina dopo la riga nera a pie di pagina in centro. Con questo rimuovo le linee nere e metto il numero di pagina in basso a destra
\pagestyle{fancy}
\fancyhead{}
\fancyfoot{}
%\fancyfoot[R]{\thepage}

\renewcommand{\headrulewidth}{0pt}
\renewcommand{\footrulewidth}{0pt}

% LaTeX funziona ad "ambienti", quello principale è il document, dove ci mettiamo il contenuto del file, tutto quello che c'è prima è solo il preambolo e serve ad impostare vari parametri
\begin{document}
  % Creò un ambiente che centra il suo contenuto
  \begin{center}
    % \Large o \small aumentano il font di quell che viene dopo, \textbf{} è il grassetto.
    % In LaTeX per andare a capo bisogna lasciare una linea vuota, oppure inserire \\ a fine riga
    \Large\textbf{Relazione Progetto Gurobi}

    \small Gruppo N° 81: Brignoli Muscio
  \end{center}\bigskip

  La prima cosa che abbiamo fatto è stato costruire la funzione obiettivo del modello e, capito che dovevamo \textit{minimizzare} lo scarto, abbiamo definito questa funzione:
  \begin{equation*}
    min\quad\left|\sum_{j=1}^{\frac{K}{2}}\sum_{i=1}^{M}P_{ij}\,x_{ij}-\sum_{j=\frac{K}{2}+1}^{K}\sum_{i=1}^{M}P_{ij}\,x_{ij}\right|
  \end{equation*}
  Dove $x_{ij}$ rappresenta la matrice delle incognite popolata per colonne.

  Essendo che la funzione presentava un modulo abbiamo creato la variabile $a$ per rappresentarla ed inserito i due vincoli che sarebbero andati a sciogliere il modulo e i restanti vincoli.

  Quindi, abbiamo costruito questo modello:
  \begin{gather*}
    min\quad a \\
    \sum_{j=1}^{\frac{K}{2}}\sum_{i=1}^{M}P_{ij}\,x_{ij}-\sum_{j=\frac{K}{2}+1}^{K}\sum_{i=1}^{M}P_{ij}\,x_{ij}\leq a \\
    -\sum_{j=1}^{\frac{K}{2}}\sum_{i=1}^{M}P_{ij}\,x_{ij}+\sum_{j=\frac{K}{2}+1}^{K}\sum_{i=1}^{M}P_{ij}\,x_{ij}\leq a \\
    \sum_{i=1}^M C_{ij}\,x_{ij}\geq \Omega\,\sum_{i=1}^M\beta_i \\
    \sum_{j=1}^{K}\sum_{i=1}^{M} P_{ij}\,x_{ij}\geq S \\
    \sum_{j=1}^K C_{ij}\,x_{ij}\leq \beta_i \\
    x_{ij}\leq \tau_{ij} \\
    x_{ij}\geq 0
  \end{gather*}

\end{document}