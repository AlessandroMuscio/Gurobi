\documentclass[12pt]{article}

\usepackage[utf8]{inputenc}
\usepackage[italian]{babel}
\usepackage[T1]{fontenc}
\usepackage[a4paper,total={180mm, 267mm},left=14mm,top=25mm]{geometry}
\usepackage{amsfonts, amssymb, fancyhdr}
\usepackage{amsmath}

\setlength{\parindent}{0pt}

\pagestyle{fancy}
\fancyhead{}
\fancyfoot{}

\renewcommand{\headrulewidth}{0pt}
\renewcommand{\footrulewidth}{0pt}

\begin{document}
\begin{center}
	\Large\textbf{Relazione Progetto Gurobi - Parte Seconda}

	\small Coppia N° 81: Brignoli Muscio
\end{center}

\section*{Quesito I}
Segue il modello formulato per il problema assegnatoci:
\begin{gather*}
	\min\sum_{i=1}^n \sum_{i\neq j,j=1}^n\; costi_{ij}x_{ij}\\
	\sum_{i=1,i\neq j}^n x_{ij}=1\qquad j=1,\dots,n;\\
	\sum_{j=1,j\neq i}^n x_{ij}=1\qquad i=1,\dots,n;\\
	1\leq u_i\leq n-1\qquad 2\leq i\leq n;\\
	u_i-u_j+(n-1)x_{ij}\leq n-2\qquad 2\leq i\neq j\leq n;\\\\
	u_i\in\mathbb{Z}\qquad i=2,\dots,n;\\
	x_{ij}\in\{0,1\}\qquad i,j=1,\dots,n.
\end{gather*}

\section*{Quesito II}
Per verificare la presenza di ulteriori soluzioni ottime, con costo uguale a quella precedentemente determinata, abbiamo impostato i seguenti parametri del modello:
\begin{itemize}
	\item `\textit{PoolSearchMode}' = 2;
	\item `\textit{SolutionNumber}' = 1.
\end{itemize}

Abbiamo quindi utilizzato il valore del parametro `\textit{PoolObjVal}' per confrontare il valore della nuova funzione obiettivo con quello precedentemente determinato , osservando che, effettivamente, questi risultano essere coincidenti; la nuova soluzione ha quindi medesimo costo e un ciclo ottimo differente.

Abbiamo deciso di inserire un controllo per verificare la corrispondenza tra i costi, in questo modo è possibile sfruttare altri set di dati ottenendo risultati coerenti.

\section*{Quesito III}
Risultano necessari i seguenti vincoli non lineari:
\begin{gather*}
	(b)\qquad n\cdot n\cdot x_{[b1][b2]}-c\cdot x_{[b1][b2]}+\sum_{i=1}^n \sum_{i\neq j,j=1}^n\;costi_{[i][j]}x_{[i][j]}\leq n\cdot n;\\\\
	(c)\qquad x_{[e1][e2]}\cdot x_{[f1][f2]}\geq x_{[d1][d2]};\\\\
	(d)\qquad \left(x_{[g1][g2]}\cdot x_{[h1][h2]}\cdot x_{[i1][i2]}\right)\cdot l=\text{costoAggiuntivo}
\end{gather*}

Trasformando in forma lineare, il modello diventa:
\begin{gather*}
	\min\sum_{i=1}^n \sum_{i\neq j,j=1}^n\; costi_{[i][j]}x_{[i][j]}+\text{costoAggiuntivo}\\
	\sum_{i=1,i\neq j}^n x_{ij}=1\qquad j=1,\dots,n;\\
	\sum_{j=1,j\neq i}^n x_{ij}=1\qquad i=1,\dots,n;\\
	1\leq u_i\leq n-1\qquad 2\leq i\leq n;\\
	u_i-u_j+(n-1)x_{ij}\leq n-2\qquad 2\leq i\neq j\leq n;\\\\
	(a)\qquad \sum_{i\neq v,i=1}^n\; costi_{[i][v]}x_{[i][v]}+\sum_{j\neq v,j=1}^n\; costi_{[v][j]}x_{[v][j]}-\frac{a}{100}\left(\sum_{i=1}^n \sum_{i\neq j,j=1}^n\; costi_{[i][j]}x_{[i][j]}\right)\leq 0;\\\\
	(b)\qquad Mx_{[b1][b2]}-c\cdot x_{[b1][b2]}+\sum_{i=1}^n \sum_{i\neq j,j=1}^n\;costi_{[i][j]}x_{[i][j]}\leq M;\\\\
	(c)\qquad p\geq x_{[d1][d2]};\\
	\qquad p\leq x_{[e1][e2]};\\
	\qquad p\leq x_{[f1][f2]};\\
	\qquad p-x_{[e1][e2]}-x_{[f1][f2]}\geq -1;\\\\
	(d)\qquad q\cdot l=\text{costoAggiuntivo};\\
	\qquad q\leq x_{[g1][g2]};\\
	\qquad q\leq x_{[h1][h2]};\\
	\qquad q\leq x_{[i1][i2]};\\
	\qquad q-x_{[g1][g2]}-x_{[h1][h2]}-x_{[i1][i2]}\geq -2.\\\\
	u_i\in\mathbb{Z}\qquad i=2,\dots,n;\\
	x_{ij}\in\{0,1\}\qquad i,j=1,\dots,n;\\
	f,q\in \{0,1\}.
\end{gather*}\bigskip

\textbf{Note}:
\begin{enumerate}
	\item Nella risposta del terzo quesito abbiamo inserito gli indici all'interno delle quadre $[\;]$ in modo da facilitarne la lettura;
	\item Utilizziamo $M$ per indicare un valore grande, per il nostro modello scegliamo: $M=n\cdot n$.
\end{enumerate}
\end{document}