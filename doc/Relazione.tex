% Il documento è un article, LaTeX ha delle classi standard che impostano già dei parametri per te, in più gli ho aumentato la grandezza al font così arriviamo prima ad una pagina ;)
\documentclass[12pt]{article}

% Imposto la codifica del file ad UTF-8
\usepackage[utf8]{inputenc}
% Imposto la lingua in italiano
\usepackage[italian]{babel}
% Questo ho capito che imposta la codifica del font ma non so cosa sia T1
\usepackage[T1]{fontenc}
% Imposto il formato, dimensioni totali, margini 
\usepackage[a4paper,total={180mm, 267mm},left=14mm,top=20mm]{geometry}
% I prime tre pacchetti ti fanno fare tutte le equazione super belle con i simboli belli, il 4° vedi dopo
\usepackage{amsfonts, amssymb, fancyhdr}
\usepackage{amsmath}

% Le impostazioni di indentazione di default di LaTeX sono strane, questo le rimuove
\setlength{\parindent}{0pt}

% Queste 7 righe sono possibili grazie al famoso 4° pacchetto, LaTeX di default mi mette delle barre nere a inizio e a pie di pagina inoltre a mettere il numero di pagina dopo la riga nera a pie di pagina in centro. Con questo rimuovo le linee nere e metto il numero di pagina in basso a destra
\pagestyle{fancy}
\fancyhead{}
\fancyfoot{}
%\fancyfoot[R]{\thepage}

\renewcommand{\headrulewidth}{0pt}
\renewcommand{\footrulewidth}{0pt}

% LaTeX funziona ad "ambienti", quello principale è il document, dove ci mettiamo il contenuto del file, tutto quello che c'è prima è solo il preambolo e serve ad impostare vari parametri
\begin{document}
  % Creò un ambiente che centra il suo contenuto
  \begin{center}
    % \Large o \small aumentano il font di quell che viene dopo, \textbf{} è il grassetto.
    % In LaTeX per andare a capo bisogna lasciare una linea vuota, oppure inserire \\ a fine riga
    \Large\textbf{Relazione Progetto Gurobi - Parte Prima}

    \small Coppia N° 81: Brignoli Muscio
  \end{center}\bigskip

  Segue il modello formulato per il problema assegnatoci:
  \begin{gather*}
    min\quad a \\
    \sum_{j=1}^{\frac{K}{2}}\sum_{i=1}^{M}P_{ij}\,x_{ij}-\sum_{j=\frac{K}{2}+1}^{K}\sum_{i=1}^{M}P_{ij}\,x_{ij}\leq a \\
    -\sum_{j=1}^{\frac{K}{2}}\sum_{i=1}^{M}P_{ij}\,x_{ij}+\sum_{j=\frac{K}{2}+1}^{K}\sum_{i=1}^{M}P_{ij}\,x_{ij}\leq a \\
    \sum_{i=1}^M C_{ij}\,x_{ij}\geq \Omega\,\sum_{i=1}^M\beta_i \\
    \sum_{j=1}^{K}\sum_{i=1}^{M} P_{ij}\,x_{ij}\geq S \\
    \sum_{j=1}^K C_{ij}\,x_{ij}\leq \beta_i \\
    x_{ij}\leq \tau_{ij} \\
    x_{ij}\geq 0
  \end{gather*}
  \hfill$\forall i,\:j\in\mathbb{N}:1\leq i\leq M\wedge1\leq j\leq K$

  \section*{Spiegazioni Quesito III}
  {\Large\bfseries Procedura 1}\\
  Come prima procedura per trovare una soluzione non ottima ammissibile al problema abbiamo preso uno dei vertici ottimi trovati e abbiamo determinando così un vettore $\boldsymbol{\lambda}$ di $q$ valori tale che \\ 
  $\boldsymbol{\lambda}={\displaystyle\sum_{i=1}^q}\lambda_i=1$ in modo da poterli così utilizzare in una \textbf{combinazione convessa} di questo vertice.\bigskip\bigskip
  
  {\Large\bfseries Procedura 2}\\
  Come seconda procedura abbiamo sfruttato il problema risolto in \textbf{forma standard} per poi valorizzare una delle slack di uno dei vincoli che identificano il vertice ottimo ad un valore non nullo maggiore di zero.\\
  Nel nostro caso specifico abbiamo scelto di valorizzare $s_3$, variabile di slack associata al primo vincolo di costo, a $100$.\bigskip\bigskip

  {\Large\bfseries Procedura 3}\\
  Come terza ed ultima procedura per trovare una soluzione non ottima ammissibile al problema abbiamo deciso di risolverlo utilizzando il metodo delle \textbf{due fasi} fermandoci, però, alla fase uno.\\
  In questo modo abbiamo ottenuto un punto da cui partire per risolverle il problema utilizzando il simplesso cioè una soluzione ammissibile non ottima.
\end{document}