% Il documento è un article, LaTeX ha delle classi standard che impostano già dei parametri per te, in più gli ho aumentato la grandezza al font così arriviamo prima ad una pagina ;)
\documentclass[12pt]{article}

% Imposto la codifica del file ad UTF-8
\usepackage[utf8]{inputenc}
% Imposto la lingua in italiano
\usepackage[italian]{babel}
% Questo ho capito che imposta la codifica del font ma non so cosa sia T1
\usepackage[T1]{fontenc}
% Imposto il formato, dimensioni totali, margini 
\usepackage[a4paper,total={180mm, 267mm},left=14mm,top=10mm]{geometry}
% I prime tre pacchetti ti fanno fare tutte le equazione super belle con i simboli belli, il 4° vedi dopo
\usepackage{amsfonts, amssymb, fancyhdr}
\usepackage{amsmath}

% Le impostazioni di indentazione di default di LaTeX sono strane, questo le rimuove
\setlength{\parindent}{0pt}

% Queste 7 righe sono possibili grazie al famoso 4° pacchetto, LaTeX di default mi mette delle barre nere a inizio e a pie di pagina inoltre a mettere il numero di pagina dopo la riga nera a pie di pagina in centro. Con questo rimuovo le linee nere e metto il numero di pagina in basso a destra
\pagestyle{fancy}
\fancyhead{}
\fancyfoot{}
%\fancyfoot[R]{\thepage}

\renewcommand{\headrulewidth}{0pt}
\renewcommand{\footrulewidth}{0pt}

% LaTeX funziona ad "ambienti", quello principale è il document, dove ci mettiamo il contenuto del file, tutto quello che c'è prima è solo il preambolo e serve ad impostare vari parametri
\begin{document}
  % Creò un ambiente che centra il suo contenuto
  \begin{center}
    % \Large o \small aumentano il font di quell che viene dopo, \textbf{} è il grassetto.
    % In LaTeX per andare a capo bisogna lasciare una linea vuota, oppure inserire \\ a fine riga
    \Large\textbf{Relazione Progetto Gurobi - Parte Seconda}

    \small Coppia N° 81: Brignoli Muscio
  \end{center}

  \section*{Quesito I}
  Segue il modello formulato per il problema assegnatoci:
  \begin{gather*}
    \min\sum_{i=1}^n \sum_{i\neq j,j=1}^n\; c_{ij}x_{ij}\\
    u_i\in\mathbb{Z}\qquad i=2,\dots,n;\\
    1\leq u_i\leq n-1\qquad 2\leq i\leq n;\\
    x_{ij}\in\{0,1\}\qquad i,j=1,\dots,n;\\
    \sum_{i=1,i\neq j}^n x_{ij}=1\qquad j=1,\dots,n;\\
    \sum_{j=1,j\neq i}^n x_{ij}=1\qquad i=1,\dots,n;\\
    u_i-u_j+(n-1)x_{ij}\leq n-2\qquad 2\leq i\neq j\leq n.
  \end{gather*}

  \section*{Quesito II}
  Per verificare la presenza di ulteriori soluzioni ottime, con costo uguale a quella precedentemente determinata, abbiamo impostato i seguenti parametri del modello:
  \begin{itemize}
    \item `\textit{PoolSearchMode}' = 2;
    \item `\textit{SolutionNumber}' = 1.
  \end{itemize}

  Abbiamo quindi utilizzato il parametro `\textit{PoolObjVal}' per confrontare il nuovo valore della funzione obiettivo con quello precedentemente determinato nel quesito 1, osservando che, effettivamente, questi risultano essere coincidenti; la nuova soluzione ha quindi medesimo costo e un ciclo ottimo differente.

  Abbiamo deciso di inserire un controllo per verificare la corrispondenza tra i costi, in questo modo è possibile sfruttare altri set di dati ottenendo risultati coerenti.

\end{document}